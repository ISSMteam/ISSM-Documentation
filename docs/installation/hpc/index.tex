% Just The Docs Front Matter
% title: High-Performance Computing (HPC)
% parent: Installation
% nav_order: 4
% has_children: false
% has_toc: false

\section{High-Performance Computing (HPC)} \label{sec:installation-hpc}
\subsection{A Note About HPC Environments}
MATLAB and Python are used only for model setup and post-processing of simulation results (e.g. plotting). As such, when leveraging the power of HPC, our general strategy is to install one copy of ISSM with the MATLAB and/or Python wrappers on a local machine, and a second copy of ISSM with only the binaries (e.g. \lstinlinebg|issm.exe|) on the cluster. We can achieve this second type of build by configuring ISSM with the \lstinlinebg|--without-wrappers| option. The MATLAB or Python interface can then send the binary input files to the cluster and fetch the output file once the run is completed.

Note as well that the `local' machine in the above case may be one that is physically remote to you. For example, you might install ISSM with its Python interface so that it is available in a remote Jupyter Lab/Hub environment. This paradigm is becoming increasingly popular with computing centers that provide access to HPC.

\subsection{Configuration and Compiling ISSM}
Please see the \href{https://issm.ess.uci.edu/trac/issm/wiki}{ISSM Development Wiki} for notes on configuring and compiling on various HPC systems.

We will be working soon to migrate the content from the wiki to this documentation.

\clearpage % Make sure all figures are placed before next section
