% Just The Docs Front Matter
% title: Parameter Files
% parent: Parameterization
% has_children: false
% has_toc: false

\subsection{Parameter Files} \label{sec:using-issm-parameterization-parameter-files}
To run a simulation, the solution sequence needs many parameters: physical constants, number of iterations, relaxation constant, thickness and surface of the glacier, etc. All of this can be done during model setup in your \lstinlinebg|runme|. But for the sake of organization and/or reusability, you might want to store the parameterization commands in a separate file. For example, for the MATLAB interface, you might create a file \lstinlinebg|Parameters.par| with contents,
\begin{lstlisting}
%%%%%%%%%%%%%%%%%%%%%%%%%%%%%%%%% GEOMETRY %%%%%%%%%%%%%%%%%%%%%%%%%%%%%%%%%

disp('      reading thicknesses');
md.geometry.thickness = InterpFromFile(md.mesh.x, md.mesh.y, thicknesspath, 10);

disp('      reading dem');
md.geometry.surface = InterpFromFile(md.mesh.x, md.mesh.y, surfacepath, 10);

%get base
md.geometry.base = md.geometry.surface - md.geometry.thickness;

%%%%%%%%%%%%%%%%%%%%%%%%%%%%%%% OBSERVATIONS %%%%%%%%%%%%%%%%%%%%%%%%%%%%%%%

disp('      reading velocities');
md = plugvelocities(md, velocitypath, 0);

disp('      loading temperature');
md.initialization.temperature = InterpFromFile(md.mesh.x, md.mesh.y, temperaturepath, 253);

disp('      creating mass balance rates');
md.smb.mass_balance = InterpFromFile(md.x, md.y, massbalancepath, 1);

disp('      loading geothermal flux'); 
load(heatfluxpath); 
md.basalforcings.geothermalflux = InterpFromGrid(x_m, y_m, heatflux, md.mesh.x, md.mesh.y, 80);

%%%%%%%%%%%%%%%%%%%%%%%%%%%%%%%%% MATERIAL %%%%%%%%%%%%%%%%%%%%%%%%%%%%%%%%%

%flow law 
disp('      creating flow law parameters');
md.materials.rheology_n = 3 * ones(md.mesh.numberofelements, 1);
md.materials.rheology_B = paterson(md.initialization.temperature);

%%%%%%%%%%%%%%%%%%%%%%%%%%% BOUNDARY CONDITIONS %%%%%%%%%%%%%%%%%%%%%%%%%%%%

%drag md.drag or stress
md.friction.coefficient = 300 * ones(md.mesh.numberofvertices, 1); %q=1.

%floating ice: no drag
md.friction.coefficient(find(md.mask.ocean_levelset < 0.)) = 0.;
md.friction.p = ones(md.mesh.numberofelements, 1);
md.friction.q = ones(md.mesh.numberofelements, 1);

%Create ice front
md = SetMarineIceSheetBC(md);
\end{lstlisting}
As you can see, even though the file extension is \lstinlinebg|par|, it is really just a MATLAB script. You can now parameterize your model in your \lstinlinebg|runme| with,
\begin{lstlisting}
>> md = parameterize(md, 'Parameters.par');
\end{lstlisting}

For the Python interface, we similarly use Pickle files (\lstinlinebg|pkl|) and parameterize with,
\begin{lstlisting}
>>> md = parameterize(md, 'Parameters.pkl')
\end{lstlisting}

%__@MARKDOWN_ONLY_START@__
%{: .highlight-title }
%> Note
%>
%> - The parameterization must be done on a two dimensional mesh.
%> - The parameters will be automatically extruded if the mesh is extruded.
%__@MARKDOWN_ONLY_END@__

%__@LATEX_ONLY_START@__
\emph{NOTE}:
\begin{itemize}
	\item Parameterization must be done on a two dimensional mesh.
	\item Parameters will be automatically extruded if the mesh is extruded.
\end{itemize}
%__@LATEX_ONLY_END@__

\clearpage % Make sure all figures are placed before next section
