% Just The Docs Front Matter
% title: Basal Friction
% parent: Parameterization
% has_children: false
% has_toc: false

\subsection{Basal Friction} \label{sec:using-issm-parameterization-friction}
\subsubsection{Introduction}
All friction laws in ISSM are implemented as:
\begin{equation}
	\boldsymbol{\tau}_b = -f\left({\bf v}_b,N\right) \frac{ {\bf v}_b}{\left|{\bf v}_b\right|}
\end{equation}
where $N$ is the effective pressure, $\boldsymbol{\tau}_b$ and ${\bf v}_b$ are the basal stress and
sliding velocities respectively. The friction laws described below are describing the norm of the
basal stress for simplicity but all implementations are such that the oppose motion (i.e. the
direction of the basal stress is the opposite of ${\bf v}_b$).

Most friction laws use a switch to define how the effective pressure, $N = p_{ice} - p_{water}$ is
calculated: \lstinlinebg|md.friction.coupling|:
\begin{itemize}
	\item 0: $p_{water} = -\rho_w g b$  uniform sheet (negative water pressure ok, default)
	\item 1: $p_{water} = 0$, so that $N=p_{ice}=\rho_i g H$ is equal to the overburden pressure
	\item 2: $p_{water} = \max\left(0,-\rho_w g b\right)$. Same as 0, but $p_{water}\ge 0$
	\item 3: Use effective pressure prescrived in \lstinlinebg|md.friction.effective_pressure|
	\item 4: Use effective pressure dynamically calculated by the hydrology model (i.e., fully
		coupled)
\end{itemize}

\subsubsection{Budd Friction law (friction)}
The default friction law is defined as \citep{Paterson1994} (p 151):
\begin{equation}
	v_b \propto N^{-q} {\tau}_b^p
\end{equation}
where:
\begin{itemize}
	\item $v_b$ is the basal velocity magnitude
	\item $\tau_b$ is the basal stress magnitude
	\item $N$ is the effective pressure
	\item $p$ and $q$ are friction law exponents
\end{itemize}

In ISSM, this friction law is implemented in terms of basal stress, following \cite{Budd1979}:
\begin{equation}
	\tau_b = C_b^2 N^r {v}_b^s
\end{equation}
where:
\begin{itemize}
	\item $C_b$ friction coefficient
	\item $r$ and $s$ are friction law exponents:
\end{itemize}
\begin{equation}
	r=q/p \hspace{4em} s=1/p
\end{equation}

This friction law can be selected as follows:
\begin{lstlisting}
>> md.friction = friction();
\end{lstlisting}

The following fields need to be specified:
\begin{itemize}
	\item \lstinlinebg|md.friction.coefficient|: friction coefficient
	\item \lstinlinebg|md.friction.p|: p exponent
	\item \lstinlinebg|md.friction.q|: q exponent
\end{itemize}

\subsubsection{Weertman Friction law (weertmanfriction)}
The Weertman friction \citep{Weertman1957} law reads:
\begin{equation}
	v_b  = C_w {\tau}_b^m
\end{equation}
\begin{itemize}
	\item $C_w$ is a friction coefficient (variable in space)
	\item $m$ is a friction law exponent
\end{itemize}

In ISSM, this friction law is implemented in terms of basal stress:
\begin{equation}
	\boldsymbol{\tau}_b = C_w^{-1/m} \|{\bf v}_b\|^{1/m-1} {\bf v}_b
\end{equation}

This friction law can be selected as follows:
\begin{lstlisting}
>> md.friction = frictionweertman();
\end{lstlisting}

One can display the following fields by running:
\begin{lstlisting}
>> md.friction
\end{lstlisting}
\begin{itemize}
	\item \lstinlinebg|md.friction.C|: friction coefficient
	\item \lstinlinebg|md.friction.m|: m exponent
\end{itemize}

\subsubsection{Coulomb-limited sliding 1 (frictioncoulomb)}
\begin{equation}
	\tau_b = \min\left(C N ub , C_c^2 N \right) 
\end{equation}

\subsubsection{Regularized Coulomb-limited sliding 1 (frictionregcoulomb)}
Sliding law from \cite{Joughin2019}:
\begin{equation}
	\tau_b = \frac{C u_b^{1/m}\alpha^2 N}{\left(\frac{u_b}{u_0} + 1\right)^{1/m}}
\end{equation}

\subsubsection{Coulomb-limited sliding 2 (frictioncoulomb2)}
Coulomb-limited sliding law used in MISMIP+ \citep{Cornford2020}:
\begin{equation}
	\tau_b = \frac{C u_b^{m}\alpha^2 N}{\left(C^{1/m}u_b + (\alpha^2N)^{1/m}\right)^{m}},
\end{equation}
where $\alpha^2 = 0.5$. Note that this friction law is exactly the same as \lstinlinebg|frictionschoof| described below, with $C_{max} = 0.5$.

\subsubsection{Regularized Coulomb-limited sliding 2 (frictionregcoulomb2)}
Sliding law from \cite{Helanow2021}:
\begin{equation}
	\tau_b = \frac{C\, N\, u_b^{1/m}}{\left(u_b + (K\,N)^{m}\right)^{1/m}}
\end{equation}

\subsubsection{Friction Tsai (frictiontsai)}
from \citep{Tsai2015}:
\begin{equation}
	\tau_b = \min\left(C ub^{m} , f N \right) 
\end{equation}

\subsubsection{Friction Schoof (frictionschoof)}
from \citep{Schoof2005,Gagliardini2007} (note that we use $C_s^2$ to make sure it is a positive number):
\begin{equation}
	\tau_b = \frac{C_s^2 v_b^{m}}{\left(1 + \left(\frac{C_s^2}{C_{max}N}\right)^{1/m} v_b\right)^{m}},
\end{equation}

\subsubsection{Friction PISM (frictionpism)}
Under construction

\subsubsection{Thin water layer friction law (frictionwaterlayer)}
The thin water layer friction law is similar to the default friction law except that the effective pressure includes a specified layer of water at the bed:
\begin{equation}
	N= g \left( \rho_i H + \rho_w \left( b - w\right) \right)
\end{equation}
when the bedrock is below sea level, and:
\begin{equation}
	N= g \left( \rho_i H - \rho_w w \right)
\end{equation}
when the bedrock is above sea level, with:
\begin{itemize}
	\item $N$ the effective pressure
	\item $\rho_i$ the ice density
	\item $\rho_w$ the water density
	\item $H$ and $b$ ice thickness and bed elevation
	\item $w$ the water thickness at the ice base
\end{itemize}

This friction law can be selected as follows:
\begin{lstlisting}
>> md.friction = frictionwaterlayer();
\end{lstlisting}

One can display all these fields by running:
\begin{lstlisting}
>> md.friction
\end{lstlisting}
\begin{itemize}
	\item \lstinlinebg|md.friction.coefficient|: friction coefficient
	\item \lstinlinebg|md.friction.p|: p exponent
	\item \lstinlinebg|md.friction.q|: q exponent
	\item \lstinlinebg|md.friction.water_layer|: thin water layer thickness (meters)
\end{itemize}

\clearpage % Make sure all figures are placed before next section
