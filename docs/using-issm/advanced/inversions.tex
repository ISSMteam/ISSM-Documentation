% Just The Docs Front Matter
% title: Inversions
% parent: Advanced Features
% has_children: false
% has_toc: false

\subsection{Inversions} \label{sec:using-issm-advanced-inversions}
\subsubsection{Introduction}%{{{
Inversions are used to constrain poorly known model parameters such as basal friction. The method consists of finding a set of model inputs that minimizes the cost function ${\mathcal J}$ that measures the misfit between model and observations. For example, inverse methods are used to infer the basal friction $k$:
\begin{equation}
	\boldsymbol{\tau}_b = -k^2 N^r \|{\bf v}\|^{s-1} {\bf v}_b
\end{equation}
and/or the depth-averaged ice hardness, $B$, in Glen's flow law:
\begin{equation}
	\mu = \frac{B}{2\left( \dot{\varepsilon}_e^{1-\frac{1}{n}}\right) }\\
\end{equation}

This section explains how to launch an inverse method and how optimization parameters must be tuned.
%}}}

\subsubsection{Cost functions}%{{{
\paragraph{Absolute misfit}%{{{
This is the classic way of calculating a misfit between a modeled and observed velocity field:
\begin{equation}
	{\mathcal J\left({\bf v}\right)}
	=\int_{S} \dfrac{1}{2}\left( 
	 \left(v_x-v_x^{\text{obs}}\right)^{2}
	+\left(v_y-v_y^{\text{obs}}\right)^{2}
	\right) dS
\end{equation}
where:
\begin{itemize}
	\item v\textsubscript{x} is the x component of the glacier modeled velocity
	\item v\textsubscript{y} is the y component of the glacier modeled velocity
	\item v\textsubscript{x}\textsuperscript{obs} is the x component of the glacier observed velocity
	\item v\textsubscript{y}\textsuperscript{obs} is the y component of the glacier observed velocity
\end{itemize}
%}}}

\paragraph{Relative misfit}%{{{
The relative misfit is defined as follows:
\begin{equation}
	{\mathcal J\left({\bf v}\right)}=
	\int_{S} \dfrac{1}{2}\left(
	 \dfrac{\left(v_x-v_x^{\text{obs}}\right)^{2}}{\left(v_x^{\text{obs}}+\varepsilon\right)^{2}}
	+\dfrac{\left(v_y-v_y^{\text{obs}}\right)^{2}}{\left(v_y^{\text{obs}}+\varepsilon\right)^{2}}
	\right) dS
\end{equation}
where:
\begin{itemize}
	\item $\varepsilon$ is a minimum velocity used to avoid the observed velocity being equal to zero.
\end{itemize}
%}}}

\paragraph{Logarithmic misfit}%{{{
\begin{equation}
	{\mathcal J\left({\bf v}\right)}
	=\int_{S} \left(\text{log}\left(
	\dfrac{\|{\bf v}\|+\varepsilon}{\|{\bf v}^{\text{obs}}\|+\varepsilon} 
	\right) \right)^2 dS
\end{equation}
where:
\begin{itemize}
	\item v is the glacier modeled velocity magnitude
	\item v\textsuperscript{obs} is the glacier observed velocity magnitude
	\item $\varepsilon$ is a minimum velocity used to avoid the observed velocity being equal to zero
\end{itemize}
%}}}

\paragraph{Thickness misfit}%{{{
\begin{equation}
	{\mathcal J\left(H\right)}
	=\int_{\Omega} \dfrac{1}{2}
	\left(H-H^{\text{obs}}\right)^{2}
	d\Omega
\end{equation}
where:
\begin{itemize}
	\item H is the ice thickness
	\item H\textsuperscript{obs} is the measured ice thickness
\end{itemize}
%}}}

\paragraph{Drag gradient}%{{{
\begin{equation}
	{\mathcal J\left(k\right)}
	=\int_{B} \gamma \dfrac{1}{2}
	\|\nabla k \|^{2}
	dB
\end{equation}
where:
\begin{itemize}
	\item $\gamma$ is a Tikhonov regularization parameter
\end{itemize}
%}}}

\paragraph{Thickness gradient}%{{{
\begin{equation}
	{\mathcal J\left(k\right)}
	=\int_{\Omega} \gamma \dfrac{1}{2}
	\|\nabla H \|^{2}
	d\Omega
\end{equation}
where:
\begin{itemize}
	\item $\gamma$ is a Tikhonov regularization parameter
\end{itemize}
%}}}
%}}}

\subsubsection{Model parameters}%{{{
The parameters relevant to the stress balance solution can be displayed by typing:
\begin{lstlisting}
>> md.inversion
\end{lstlisting}

\begin{itemize}
	\item \lstinlinebg|md.inversion.iscontrol|: 1 if inversion is activated, 0 for a forward run (default)
	\item \lstinlinebg|md.inversion.incomplete_adjoint|: 1 linear viscosity, 0 non-linear viscosity
	\item \lstinlinebg|md.inversion.control_parameters|: parameters that are inferred (ex: \lstinlinebg|{'FrictionCoefficient'}| or \lstinlinebg|{'MaterialsRheologyBbar'}|
	\item \lstinlinebg|md.inversion.cost_functions|: list of individual cost functions that are summed to calculate the final cost function $\mathcal J$ to be minimized (ex: \lstinlinebg|[101, 501]|)
	\item \lstinlinebg|md.inversion.cost_functions_coefficients|: weight of each individual cost function previously defined for each vertex (more/no weight can be put on certain regions)
	\item \lstinlinebg|md.inversion.min_parameters|: minimum value for the inferred parameter
	\item \lstinlinebg|md.inversion.max_parameters|: maximum value for the inferred parameter
	\item \lstinlinebg|md.inversion.vx_obs|: x component of the surface velocity
	\item \lstinlinebg|md.inversion.vy_obs|: y component of the surface velocity
	\item \lstinlinebg|md.inversion.vel_obs|: surface velocity magnitude
	\item \lstinlinebg|md.inversion.thickness_obs|: measured ice thickness
\end{itemize}
%}}}

\subsubsection{Minimization algorithms}%{{{
Depending on the class of \lstinlinebg|md.inversion|, several optimization algorithm are available:
\begin{itemize}
	\item Brent search algorithm (\lstinlinebg|md.inversion = inversion()|, the default)
	\item Toolkit for Advanced Optimization (TAO) (\lstinlinebg|md.inversion = taoinversion()|)
	\item M1QN3 algorithm (\lstinlinebg|md.inversion = m1qn3inversion()|)
\end{itemize}
Each minimizer has its own optimization parameters described below.

\paragraph{Brent search minimizers}%{{{
\begin{itemize}
	\item \lstinlinebg|md.inversion.nsteps|: number of optimization searches (gradient evaluations)
	\item \lstinlinebg|md.inversion.maxiter_per_step|: maximum iterations during each optimization step
	\item \lstinlinebg|md.inversion.step_threshold|: decrease threshold for next step (default is 30\%)
	\item \lstinlinebg|md.inversion.gradient_scaling|: scaling factor on gradient direction during optimization, for each optimization step
\end{itemize}
\begin{equation}
	\alpha\in\left[0,\text{\ttfamily gradient\_scaling} \right]\hspace{3em}
	p^{\text{new}}=p^{\text{old}}-\alpha\;\nabla_p {\mathcal J}/\|\nabla_p {\mathcal J}\|
\end{equation}
%}}}

\paragraph{Toolkit for Advanced Optimization (TAO)}%{{{
ISSM has an interface to the Toolkit for Advanced Optimization (TAO) \citep{Munson2012}. Here is a list of the relevant parameters:
\begin{itemize}
	\item \lstinlinebg|md.inversion.maxsteps|: maximum number of iterations (gradient computation)
	\item \lstinlinebg|md.inversion.maxiter|: maximum number of Function evaluation (forward run)
	\item \lstinlinebg|md.inversion.algorithm|: inimization algorithm. ex: \lstinlinebg|'tao_blmvm'|, \lstinlinebg|'tao_cg'|, \lstinlinebg|'tao_lmvm'|
	\item \lstinlinebg|md.inversion.fatol|: cost function absolute convergence criterion (defined below)
	\item \lstinlinebg|md.inversion.frtol|: cost function relative convergence criterion (defined below)
	\item \lstinlinebg|md.inversion.gatol|: gradient absolute convergence criterion (defined below)
	\item \lstinlinebg|md.inversion.grtol|: gradient relative convergence criterion (defined below)
	\item \lstinlinebg|md.inversion.gttol|: gradient relative convergence criterion 2 (defined below)
\end{itemize}
with the following convergence criteria:
\begin{equation}
	\begin{array}{lcl}
		f(X) - f(X^*)                                 & < & \epsilon_{fatol} \\
		\left|f(X) - f(X^*\right|/\left|f(X^*)\right| & < & \epsilon_{frtol} \\
		\|g(X)\|                                      & < & \epsilon_{gatol} \\
		\|g(X)\|/\left|f(X)\right|                    & < & \epsilon_{grtol} \\
		\|g(X)\|/\|g(X_0)\|                           & < & \epsilon_{gttol} \\
	\end{array}
\end{equation}
where:
\begin{itemize}
	\item $f(X)$ is the cost function at $X$
	\item $g(X)$ is the cost function gradient with respect to $X$
	\item $X^*$ is the estimated "true" minimum
	\item $X_0$ is the initial guess
\end{itemize}
%}}}

\paragraph{M1QN3}%{{{
ISSM has an interface to M1QN3 (Inria) \citep{Gilbert1989}. This interface was largely based on \cite{Nardi2009}. Here is a list of the relevant parameters:
\begin{itemize}
	\item \lstinlinebg|md.inversion.maxsteps|: maximum number of iterations (gradient computation)
	\item \lstinlinebg|md.inversion.maxiter|: maximum number of Function evaluation (forward run)
	\item \lstinlinebg|md.inversion.dxmin|:  convergence criterion: two points less than dxmin from each other (sup-norm) are considered identical
	\item \lstinlinebg|md.inversion.gttol|: gradient relative convergence criterion 2 (defined below)
\end{itemize}
%}}}
%}}}

\subsubsection{Running an inversion}
To launch an inversion, run a stress balance solution with \lstinlinebg|md.inversion.iscontrol = 1|:
\begin{lstlisting}
>> md = solve(md, 'Stressbalance');
\end{lstlisting}

\clearpage % Make sure all figures are placed before next section
