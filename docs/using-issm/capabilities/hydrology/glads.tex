% Just The Docs Front Matter
% title: Hydrology Solution - GlaDS
% parent: Hydrology Solution
% has_children: false
% has_toc: false

\subsubsection{Hydrology Solution - GlaDS} \label{sec:using-issm-capabilities-hydrology-glads}
\paragraph{Description}
The two-dimensional Glacier Drainage System model (GlaDS, \cite{Werder2013}) couples a distributed water sheet model -- a continuum description of a linked cavity drainage system \citep{Hewitt2011} -- with a channelized water flow model -- modeled as R channels \citep{Rothlisberger1972,Nye1976}. The coupled system collectively describes the evolution of hydraulic potential $\phi$, water sheet thickness $h$, and water channel cross-sectional area $S$. 

\subparagraph{Sheet model equations}
\begin{itemize}
	\item Mass conservation: The mass conservation equation describes water storage changes over longer timescales (dictated by cavity opening due to sliding) as well as shorter timescales (e.g. due to surface melt water input):
	\begin{equation}
		\frac{e_v}{\rho_w g}\frac{\partial\phi}{\partial t} + \frac{\partial h}{\partial t} - \nabla\cdot\boldsymbol{q} - m_b = 0,
	\end{equation}
	where: $e_v$ is the englacial void ratio, $\rho_w$ is water density (kg~m\textsuperscript{-3}), $g$ is gravitational acceleration (kg~m\textsuperscript{-3}), $\phi$ is the hydraulic potential (Pa), and $h$ is the sheet thickness (m). The water discharge $\boldsymbol{q}$ (m\textsuperscript{2}~s\textsuperscript{-1}) is given by:
	\begin{equation}
		\boldsymbol{q}=-k_s\;h^{\alpha_s}\left|\nabla\phi\right|^{\beta_s-2}\nabla\phi,
	\end{equation}
	where $k_s$ is the sheet conductivity (m~s~kg\textsuperscript{-1}), and $\alpha_s$$=$5/4 and $\beta_s$$=$3/2 are two constant exponents. Finally, the melt source term $m_b$ (m~s\textsuperscript{-1}) is given by:
	\begin{equation}
		m_b=\frac{G+|\boldsymbol{\tau}_b\cdot\boldsymbol{u}_b|}{\rho_{i}L},
	\end{equation}
	where $G$ is the geothermal heat flux (W~m\textsuperscript{-2}), $|\boldsymbol{\tau}_b\cdot\boldsymbol{u}_b|$ is the frictional heating (W~m\textsuperscript{-2}), for basal stress $\boldsymbol{\tau}_b$ (Pa), $\rho_i$ is ice density (kg~m\textsuperscript{-3}), and $L$ is latent heating (J~kg\textsuperscript{-1}).
\end{itemize}

\begin{itemize}
	\item Sheet thickness:
	\begin{equation}
		\frac{\partial h}{\partial t} = w_s - v_s.
	\end{equation}
	Here, $w_s$ is the cavity opening rate due to sliding over bed topography (m~s\textsuperscript{-1}), given by:
	\begin{equation}
		w_s\left(h\right) =
		\begin{array}{ll}
			\displaystyle \frac{\left|\boldsymbol{u}_b\right|}{l_r}\left(h_r-h\right), & \text{if } h<h_r\\
			\\
			0, & \text{otherwise,}
		\end{array}
	\end{equation}
	where $h_r$ and $l_r$ are both constants (m), and $\boldsymbol{u}_b$ is the basal sliding velocity vector (provided by the ice flow model). The cavity closing rate due to ice creep $v_s$ (m~s\textsuperscript{-1}), is given by:
	\begin{equation}
		v_s\left(h,\phi\right) = \frac{2A}{n^n}h\left|N\right|^{n-1}N,
	\end{equation}
	where $A$ is the basal flow parameter (Pa\textsuperscript{-3}~s\textsuperscript{-1}), related to the ice hardness by $B=A$\textsuperscript{-1/3}, $n$ is the Glen flow relation exponent, and $N= \phi_0-\phi$ is the effective pressure. The overburden hydraulic potential is given by $\phi_0 = \phi_m+p$, with the ice pressure $p = \rho_i g H$ and elevation potential $\phi_m = \rho_w g b$, all of which are given in units of Pa.
\end{itemize}

\subparagraph{Channel model equations}
\begin{itemize}
	\item Channel discharge (along mesh edges):
	\begin{equation}
		\frac{\partial Q}{\partial s} + \frac{\Xi-\Pi}{L}\left(\frac{1}{\rho_i} - \frac{1}{\rho_w}\right) - v_c - m_c = 0,
	\end{equation}
	where $Q$ is the channel discharge (m\textsuperscript{3}~s\textsuperscript{-1}), which evolves with respect to the horizontal coordinate along the channel $s$, $\Xi$ is the channel potential energy dissipated per unit length and time (W~m\textsuperscript{-1}), $\Pi$ is the channel pressure melting/refreezing (W~m\textsuperscript{-1}), $v_c$ is the channel closing rate (m\textsuperscript{2}~s\textsuperscript{-1}) and $m_c$ is the source term (m\textsuperscript{2}~s\textsuperscript{-1}). The discharge $Q$ is defined as:
	\begin{equation}
		Q= -\underbrace{k_cS^{\alpha_c}\left|\frac{\partial\phi}{\partial s}\right|^{\beta_c-2}}_{K_c}
		\frac{\partial\phi}{\partial s},
	\end{equation}
	where $k_c$ is the channel conductivity, and $\alpha_c$$=$3 and $\beta_c$$=$2 are two exponents. The term $v_c$ is the closing of the channels by ice creep, and is given by:
	\begin{equation}
		v_c\left(S,\phi\right) = \frac{2A}{n^n}S\left|N\right|^{n-1}N,\\
	\end{equation}
	where $S$ is the channel cross-sectional area (m\textsuperscript{2}). Finally, $m_c$, the channel source term balancing the flow of water out of the adjacent sheet, is: 
	\begin{equation}
		m_c = \boldsymbol{q}\cdot\boldsymbol{n}|_{\partial{\Omega_{i1}} } +
		\boldsymbol{q}\cdot\boldsymbol{n}|_{\partial{\Omega_{i2}} }.
	\end{equation}
	where $\boldsymbol{n}$ is the normal to the channel edge.
\end{itemize}

\begin{itemize}
	\item Channel dissipation of potential energy:
	\begin{equation}
		\Xi(S,\phi)=\Biggl|Q\frac{\partial\phi}{\partial s}\Biggr| +
		\left|l_cq_c\frac{\partial\phi}{\partial s}\right|,
	\end{equation}
	where $l_c$ is the channel width (m), and $q_c$ is the discharge in the sheet (flowing in the direction of the channel; m\textsuperscript{2}~s\textsuperscript{-1}), given by:
	\begin{equation}
		q_c\left(h,\phi\right)
		= -\underbrace{k_s h^{\alpha_s} \left|\frac{\partial\phi}{\partial s}\right|^{\beta_s-2}}_{K_s}
		\frac{\partial\phi}{\partial s},
	\end{equation}
	with $k_s$, $\alpha_s$, and $\beta_s$ as given above.
\end{itemize}

\begin{itemize}
	\item Channel melt/refreeze rate:
	\begin{equation}
		\Pi(S,\phi)=-c_tc_w\rho_w(Q+f l_c q_c)\frac{\partial\phi-\partial\phi_m}{\partial s},
	\end{equation}
	Here, $c_t$ is the Clapeyron slope (K~Pa\textsuperscript{-1}), $c_w$ is the specific heat capacity of water (J~kg\textsuperscript{-1}~K\textsuperscript{-1}), and $f$ is a switch parameter that accounts for the fact that the channel area cannot be negative, turning off the sheet flow for refreezing as $S\rightarrow0$, i.e.:
	\begin{equation}
		f = 
		\left\{
		\begin{array}{ll}
			1, & \text{if }S>0 \text{ or } q_c\partial(\phi-\phi_m)\partial s>0\\
			0, & \text{otherwise}
		\end{array}\right.
	\end{equation}
\end{itemize}

\begin{itemize}
	\item Cross-sectional channel area (defined along mesh edges):
	\begin{equation}
		\frac{\partial S}{\partial t} = \frac{\Xi - \Pi}{\rho_i L} - v_c.
	\end{equation}
\end{itemize}

\subparagraph{Boundary conditions}
Boundary conditions for the evolution of hydraulic potential $\phi$ are applied on the domain boundary $\partial\Omega$, as either a prescribed pressure or water flux. The Dirichlet boundary condition is:
\begin{equation}
	\phi=\phi_D  \quad\text{on} \quad\partial\Omega_D,
\end{equation}
where $\phi_D$ is a specific potential, and the Neumann boundary condition is:
\begin{equation}
	\frac{\partial\phi}{\partial n}=\Phi_N  \quad\text{on} \quad\partial\Omega_N,
\end{equation}
corresponding to the specific discharge
\begin{equation}
	q_N=-k_s h^{\alpha_s}|\nabla\phi|^{\beta_s-2}\Phi_N.
\end{equation}
Channels are defined only on element edges and are not allowed to cross the domain boundary, so we do not require flux conditions. Since the evolution equations for $h$ and $S$ do not contain their spatial derivatives, we do not require any boundary conditions for their evolution equations.

\paragraph{Model parameters}
The parameters relevant to the GlaDS (\lstinlinebg|hydrologyglads|) solution can be displayed by running:
\begin{lstlisting}
>> md.hydrology
\end{lstlisting}

\begin{itemize}
	\item \lstinlinebg|md.hydrology.pressure_melt_coefficient|: Pressure melt coefficient ($c_t$) [K Pa\textsuperscript{-1}]
	\item \lstinlinebg|md.hydrology.sheet_conductivity|: sheet conductivity ($k$) [m\textsuperscript{7/4} kg\textsuperscript{-1/2}]
	\item \lstinlinebg|md.hydrology.cavity_spacing|: cavity spacing ($l_r$) [m]
	\item \lstinlinebg|md.hydrology.bump_height|: typical bump height ($h_r$) [m]
	\item \lstinlinebg|md.hydrology.ischannels|: Do we allow for channels? 1: yes, 0: no
	\item \lstinlinebg|md.hydrology.channel_conductivity|: channel conductivity ($k_c$) [m\textsuperscript{3/2} kg\textsuperscript{-1/2}]
	\item \lstinlinebg|md.hydrology.spcphi|: Hydraulic potential Dirichlet constraints [Pa]
	\item \lstinlinebg|md.hydrology.neumannflux|: water flux applied along the model boundary (m\textsuperscript{2} s\textsuperscript{-1})
	\item \lstinlinebg|md.hydrology.moulin_input|: moulin input ($Q_s$) [m\textsuperscript{3} s\textsuperscript{-1}]
	\item \lstinlinebg|md.hydrology.englacial_void_ratio|: englacial void ratio ($e_v$)
	\item \lstinlinebg|md.hydrology.requested_outputs|: additional outputs requested?
	\item \lstinlinebg|md.hydrology.melt_flag|: User specified basal melt? 0: no (default), 1: use \lstinlinebg|md.basalforcings.groundedice_melting_rate|
\end{itemize}

\paragraph{Running a simulation}
To run a transient standalone subglacial hydrology simulation, use the following commands:
\begin{lstlisting}
md.transient = deactivateall(md.transient);
md.transient.ishydrology = 1;

%Change hydrology class to GlaDS;
md.hydrology = hydrologyglads();

%Set model parameters here;
md = solve(md, 'Transient');
\end{lstlisting}

\clearpage % Make sure all figures are placed before next section
