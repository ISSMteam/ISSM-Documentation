% Just The Docs Front Matter
% title: Hydrology Solution - SHAKTI
% parent: Hydrology Solution
% has_children: false
% has_toc: false

\subsubsection{Hydrology Solution - SHAKTI} \label{sec:using-issm-capabilities-hydrology-shakti}
\paragraph{Description}
SHAKTI (Subglacial Hydrology and Kinetic, Transient Interactions) is a transient subglacial hydrology model that has flexible geometry and treats the entire domain with one set of governing equations, allowing for any drainage configuration to arise, which can include efficient (channelized) and inefficient (distributed) features. \citep{Sommers2018}

\subparagraph{Equations}
The SHAKTI model is based upon governing equations that describe conservation of water mass, evolution of the system geometry, basal water flux (approximate momentum equation), and internal melt generation (approximate energy equation).

\begin{itemize}
	\item Continuity equation (water mass balance):
	\begin{equation}\frac{\partial b}{\partial t}+\frac{\partial b_e}{\partial t}+\nabla \cdot \vec{q}=\frac{\dot m}{\rho_w}+i_{e\rightarrow b}\end{equation}
	where $b$ is subglacial gap height, $b_e$ is the volume of water stored englacially per unit area of bed, $\vec{q}$ is basal water flux, $\dot{m}$ is melt rate, and $i_{e\rightarrow b}$ is the input rate of water from the englacial to subglacial system.
\end{itemize}

\begin{itemize}
	\item Basal gap dynamics (subglacial geometry):
	\begin{equation}\frac{\partial b}{\partial t}=\frac{\dot{m}}{\rho_i}+\beta u_b-A|p_i-p_w|^{n-1}(p_i-p_w)b\end{equation}
	where $b$ is the subglacial gap height, $\dot{m}$ is melt rate, $A$ is the ice flow law parameter, $n$ is the flow law exponent, $p_i$ is the overburden pressure of ice, $p_w$ is water pressure, $\beta$ is a dimensionless parameter governing opening by sliding, and $u_b$ is sliding velocity. According to this equation, the subglacial gap height evolves with time by: opening by both melt and sliding over bumps on the bed, and closing due to ice creep.
\end{itemize}

\begin{itemize}
	\item Basal water flux (approximate momentum equation):
	\begin{equation}\vec{q}=\frac{-b^3g}{12\nu(1+\omega Re)}\nabla h\end{equation}
	where $b$ is subglacial gap height, $g$ is gravitational acceleration, $\nu$ is kinematic viscosity of water, $\omega$ is a dimensionless parameter controlling the nonlinear transition from laminar to turbulent flow (for turbulent flow, the flux becomes proportional to the square root of the head gradient), $Re$ is the Reynolds number, and $h$ is hydraulic head:
	\begin{equation}h=\frac{p_w}{\rho_w g}+z_b\end{equation}
\end{itemize}

Equation (3) is a key piece of our model formulation, in that it allows for a spatially and temporally variable hydraulic transmissivity in the system, and facilitates easeful transition between laminar and turbulent flow regimes. Most existing models prescribe a hydraulic conductivity and assume the flow to be turbulent everywhere. Our momentum equation acts to "unify" different drainage modes in a single model.

\begin{itemize}
	\item Internal melt generation (energy balance at the bed):
	\begin{equation}
		\dot{m}=\frac{1}{L}(G+\vec{u_b}\cdot\vec{\tau_b}-\rho_wg\vec{q}\cdot\nabla h-c_t c_w \rho_w \vec{q}\cdot\nabla p_w)
	\end{equation}
	where $L$ is latent heat of fusion of water, $G$ is geothermal flux, $u_b$ is basal sliding velocity, $\tau_b$ is the stress exerted by the bed onto the ice, $\vec{q}$ is basal water flux (discharge), $h$ is hydraulic head, $c_t$ is the pressure melt coefficient (Clapeyron constant), $c_w$ is the heat capacity of water, $\rho_w$ is density of water, and $p_w$ is water pressure. In words, melt is produced through a combination of geothermal flux, frictional heat due to sliding, and heat generated through internal dissipation (where mechanical energy is converted to thermal energy), minus the heat consumed due to changes in water pressure. We note that this form of the energy equation assumes that all heat produced is converted locally to melt and is not advected downstream. We assume that the ice and liquid water are consistently at the pressure melting point temperature. While these assumptions may not be strictly valid under certain real conditions that may have interesting implications, we leave that discussion for future work.
\end{itemize}

Following Werder et al. (2013), the englacial storage volume is a function of water pressure:
\begin{equation}
	b_e=e_v\frac{\rho_wgh-\rho_wgz_b}{\rho_wg}=e_v(h-z_b)
\end{equation}
where $e_v$ is the englacial void ratio (zero for no englacial storage).

Equations (1), (2), (3), and (5) are combined to form a nonlinear partial differential equation (PDE) in terms of hydraulic head, $h$:
\begin{equation}
	\nabla\cdot\left[\frac{-b^3g}{12\nu(1+\omega Re)}\cdot\nabla h\right]+\frac{\partial e_v(h-z_b)}{\partial t}=\dot{m}\left[\frac{1}{\rho_w}-\frac{1}{\rho_i}\right]+A|p_i-p_w|^{n-1}(p_i-p_w)b-\beta u_b+i_{e\rightarrow b}
\end{equation}
With no englacial storage ($e_v=0$), Eq. (7) takes the form of an elliptic PDE.

Defining the hydraulic "transmissivity":
\begin{equation}
	\vec{K}=\frac{-b^3g}{12\nu(1+\omega Re)}
\end{equation}

Equation (7) can be written more compactly as:
\begin{equation}
	\nabla\cdot(\vec{K}\cdot\nabla h)+\frac{\partial e_v(h-z_b)}{\partial t}=\dot{m}(\frac{1}{\rho_w}-\frac{1}{\rho_i})+A|p_i-p_w|^{n-1}(p_i-p_w)b-\beta u_b+i_{e\rightarrow b}
\end{equation}
or simply as:
\begin{equation}
	\nabla\cdot(\vec{K}\cdot\nabla h)+\frac{\partial e_v(h-z_b)}{\partial t}=RHS
\end{equation}
where the forcing ($RHS$) is a function of the relevant dependent variables. Within each time step, this nonlinear PDE is solved using a Picard iteration to establish the head ($h$) distribution.

\subparagraph{Boundary conditions}
Boundary conditions can be applied as either prescribed head (Dirichlet) conditions or as flux (Neumann) conditions. We typically apply a Dirichlet boundary condition of atmospheric pressure at the edge of the ice sheet, and Neumann boundary conditions (no flux or prescribed flux, which can be constant or time-varying) on the other boundaries of the subglacial drainage domain.

\paragraph{Model parameters}
The parameters relevant to the SHAKTI (hydrologyshakti) solution can be displayed by running:
\begin{lstlisting}
>> md.hydrology
\end{lstlisting}

\begin{itemize}
	\item \lstinlinebg|md.hydrology.head| subglacial hydrology water head (m)
	\item \lstinlinebg|md.hydrology.gap_height| height of gap separating ice to bed (m)
	\item \lstinlinebg|md.hydrology.bump_spacing| characteristic bedrock bump spacing (m)
	\item \lstinlinebg|md.hydrology.bump_height| characteristic bedrock bump height (m)
	\item \lstinlinebg|md.hydrology.englacial_input| liquid water input from englacial to subglacial system (m/yr)
	\item \lstinlinebg|md.hydrology.moulin_input| liquid water input from moulins (at the vertices) to subglacial system (m\textsuperscript{3}/s)
	\item \lstinlinebg|md.hydrology.reynolds| Reynolds' number
	\item \lstinlinebg|md.hydrology.neumannflux| water flux applied along the model boundary (m\textsuperscript{2}/s)
	\item \lstinlinebg|md.hydrology.spchead| water head constraints (NaN means no constraint) (m)
	\item \lstinlinebg|md.hydrology.relaxation| under-relaxation coefficient for nonlinear iteration
	\item \lstinlinebg|md.hydrology.storage| englacial storage coefficient (void ratio)
\end{itemize}

\paragraph{Running a simulation}
To run a transient stand-alone subglacial hydrology simulation, use the following commands:
\begin{lstlisting}
md.transient = deactivateall(md.transient);
md.transient.ishydrology = 1;

%Change hydrology class to SHAKTI
md.hydrology = hydrologyshakti();

%Set model paramters here

md = solve(md, 'Transient');
\end{lstlisting}

\clearpage % Make sure all figures are placed before next section
