% Just The Docs Front Matter
% title: Hydrology Solution - Shreve Approximation
% parent: Hydrology Solution
% has_children: false
% has_toc: false

\subsubsection{Hydrology Solution - Shreve Approximation} \label{sec:using-issm-capabilities-hydrology-shreve}

\paragraph{Physical basis}
This model is the one described in \cite{LeBrocq2009}. Here we present only the main equations.

\subparagraph{Water column}
The model applied here is the most simplistic form of the water-film model, as described by the Weertman theory \citep{Weertman1957}. The model solves for the thickness $w$ of the water-film as follows:
\begin{equation}
	\frac{\partial w}{\partial t}=S - \nabla \cdot {\bf u}_{w} w
\end{equation}
where:
\begin{itemize}
	\item $S$ is the source term $[m\,s^{-1}]$
	\item ${\bf u}_w$ is the water velocity vector $[m\,s^{-1}]$
\end{itemize}

The water velocity vector ${\bf u}_w$ is a depth-averaged two dimensional horizontal vector, which is computed using a theoretical treatment of laminar flow between two parallel plates:
\begin{equation}
	{\bf u}_w = \frac{w^2}{12 \mu}\nabla \phi
\end{equation}
\begin{itemize}
	\item $\phi$ is the hydraulic potential $[Pa]$
	\item $\mu$ is the water viscosity $[Pa\,s]$
\end{itemize}

In this model, the hydraulic potential $\phi$ is defined following the Shreve approximation \citep{Shreve1972}, which hypothesizes a null effective pressure. Assuming this null effective pressure gives the hydraulic potential gradient as follows:
\begin{equation}
	\nabla \phi=\rho_{ice} g \nabla s + \left(\rho_w - \rho_{ice}\right) g \nabla h 
\end{equation}
where:
\begin{itemize}
	\item $\rho_{ice}$ is the density of the ice $[kg\,m^{-3}]$
	\item $\rho_w$ is the density of fresh water $[kg\,m^{-3}]$
	\item $s$ is the surface elevation $[m]$
	\item $g$ is the gravitational acceleration $[m\,s^{-2}]$
	\item $h$ is the bedrock elevation $[m]$
\end{itemize}

\subparagraph{Numerical implementation}
To stabilize the equation, artificial diffusion might be added to the left hand side:
\begin{equation}
	\frac{\partial w}{\partial t} +{\color{red} \nabla \left(\mathfrak{D} \nabla w\right)} =S - \nabla \cdot {\bf u}_w w
\end{equation}
where $\mathfrak{D}$ is the artificial diffusivity. We take:
\begin{equation}
	\mathfrak{D} = \frac{h}{2}
	\left(\begin{array}{cc}
		\left|vx\right| & 0 \\
		\\
		0 & \left|vy\right|
	\end{array}\right)
\end{equation}

\paragraph{Model parameters}
The parameters relevant to the water column solution can be displayed by running:
\begin{lstlisting}
>> md.hydrology
\end{lstlisting}

\begin{itemize}
	\item \lstinlinebg|md.hydrology.spcwatercolumn|: water thickness constraints (\lstinlinebg|NaN| means no constraint) $[m]$
	\item \lstinlinebg|md.hydrology.stabilization|: artificial diffusivity (default is 1).
\end{itemize}

\paragraph{Running a simulation}
To run a simulation, use the following command:
\begin{lstlisting}
>> md = solve(md, 'Hydrology');
\end{lstlisting}

\clearpage % Make sure all figures are placed before next section
