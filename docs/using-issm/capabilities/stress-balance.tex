% Just The Docs Front Matter
% title: Stress Balance Solution
% parent: Capabilities
% has_children: false
% has_toc: false

\subsection{Stress Balance Solution} \label{sec:using-issm-capabilities-stress-balance}

\subsubsection{Physical basis}
\paragraph{Conservation of linear momentum}
The conservation of momentum reads:
\begin{equation}
	\rho \frac{D {\bf v} }{Dt} = \nabla \cdot {\boldsymbol{\sigma}} + \rho {\bf b}
\end{equation}
where:
\begin{itemize}
	\item $\rho$ is the ice density
	\item ${\bf v}$ is the velocity vector
	\item $\boldsymbol{\sigma}$ is the Cauchy stress tensor
	\item ${\bf b}$ is a body force
\end{itemize}
Now if we assume that:
\begin{itemize}
	\item The ice motion is a Stokes flow (acceleration negligible)
	\item The only body force is due to gravity (Coriolis effect negligible)
\end{itemize}
The equation of momentum conservation is reduced to:
\begin{equation}
	\nabla \cdot \boldsymbol{\sigma} + \rho {\bf g} = {\bf 0}
\end{equation}
\paragraph{Conservation of angular momentum}
For a non-polar material body, the balance of angular momentum imposes the stress tensor to be symmetrical:
\begin{equation}
	\boldsymbol{\sigma} = \boldsymbol{\sigma}^T
\end{equation}
\paragraph{Ice constitutive equations}
Ice is treated as a purely viscous incompressible material \cite{Cuffey2010}. Its constitutive equation therefore only involves the deviatoric stress and the strain rate tensor:
\begin{equation}
	\boldsymbol{\sigma}' = 2\,\mu\dot{\boldsymbol{\varepsilon}}
\end{equation}
where:
\begin{itemize}
	\item $\boldsymbol{\sigma}'$ is the deviatoric stress tensor ($\boldsymbol{\sigma}' = \boldsymbol{\sigma} + p {\bf I}$)
	\item $\mu$ is the ice effective viscosity
	\item $\dot{\boldsymbol{\varepsilon}}$ is the strain rate tensor
\end{itemize}
Ice is a non-Newtonian fluid, its viscosity follows the generalized Glen's flow law \cite{Glen1955}:
\begin{equation}
	\mu = \frac{B}{2\,\dot{\varepsilon}_e^{\frac{n-1}{n}}}
\end{equation}
where:
\begin{itemize}
	\item $B$ is the ice hardness or rigidity
	\item $n$ is Glen's flow law exponent, generally taken as equal to 3
	\item $\dot{\varepsilon}_e$ is the effective strain rate
\end{itemize}
The effective strain rate is defined as:
\begin{equation}
	\dot{\varepsilon}_e = \sqrt{\frac{1}{2} \sum_{i,j} \dot{\varepsilon}_{ij}^2}= \frac{1}{\sqrt{2}} \|\dot{\boldsymbol{\varepsilon}}\|_F
\end{equation}
where $\|\cdot\|_F$ is the Frobenius norm.

\paragraph{Full-Stokes (FS) field equations}
Without any further approximation, the previous system of equations are called the \emph{Full-Stokes} model.

\paragraph{Higher-Order (HO) field equations}
We make two assumptions:
\begin{enumerate}
	\item Bridging effects are neglected
	\item Horizontal gradient of vertical velocities are neglected compared to vertical gradients of horizontal velocities
\end{enumerate}
With these two assumptions, the Full-Stokes equations are reduced to a system of 2 equations with 2 unknowns \cite{Blatter1995, Pattyn2003}:
\begin{equation}
	\begin{array}{l}
		\nabla\cdot\left(2\mu\dot{\boldsymbol{\varepsilon}}_{HO1}\right) = \rho g \dfrac{\partial s}{\partial x} \\
		\\
		\nabla\cdot\left(2\mu\dot{\boldsymbol{\varepsilon}}_{HO2}\right) = \rho g \dfrac{\partial s}{\partial y}
	\end{array}
\end{equation}
with:
\begin{equation}
	\begin{array}{l}
		\dot{\boldsymbol{\varepsilon}}_{HO1} = \left[
		\begin{array}{c}
			2\dfrac{\partial v_x}{\partial x} +  \dfrac{\partial v_y}{\partial y}\\
			\\
			\dfrac{1}{2}\left(\dfrac{\partial v_x}{\partial y} + \dfrac{\partial v_y}{\partial x}\right)\\
			\\
			\dfrac{1}{2}\dfrac{\partial v_x}{\partial z}
		\end{array}
		\right]
		\quad
		\dot{\boldsymbol{\varepsilon}}_{HO2} =\left[
		\begin{array}{c}
			\dfrac{1}{2}\left(\dfrac{\partial v_x}{\partial y} + \dfrac{\partial v_y}{\partial x}\right)\\
			\\
			\dfrac{\partial v_x}{\partial x} + 2\dfrac{\partial v_y}{\partial y}\\
			\\
			\dfrac{1}{2}\dfrac{\partial v_y}{\partial z}
		\end{array}
		\right]
	\end{array}
\end{equation}

\paragraph{Shelfy-Stream Approximation (SSA) field equations}
We make the following assumption:
\begin{enumerate}
	\item Vertical shear is negligible
\end{enumerate}
With this assumption, we have a system of 2 equations with 2 unknowns in the horizontal plane
\citep{Morland1987a, MacAyeal1989}:
\begin{equation}
	\begin{array}{l}
		\nabla\cdot\left(2\bar{\mu}H\dot{\boldsymbol{\varepsilon}}_{SSA1}\right) - \alpha^2 v_x= \rho g H \dfrac{\partial s}{\partial x} \\
		\\
		\nabla\cdot\left(2\bar{\mu}H\dot{\boldsymbol{\varepsilon}}_{SSA2}\right) - \alpha^2 v_y= \rho g H \dfrac{\partial s}{\partial y}
	\end{array}
\end{equation}
with:
\begin{equation}
	\begin{array}{l}
		\dot{\boldsymbol{\varepsilon}}_{SSA1} = \left[
		\begin{array}{c}
			2\dfrac{\partial v_x}{\partial x} +  \dfrac{\partial v_y}{\partial y}\\
			\\
			\dfrac{1}{2}\left(\dfrac{\partial v_x}{\partial y} + \dfrac{\partial v_y}{\partial x}\right)
		\end{array}
		\right]
		\quad
		\dot{\boldsymbol{\varepsilon}}_{SSA2} =\left[
		\begin{array}{c}
			\dfrac{1}{2}\left(\dfrac{\partial v_x}{\partial y} + \dfrac{\partial v_y}{\partial x}\right)\\
			\\
			\dfrac{\partial v_x}{\partial x} + 2\dfrac{\partial v_y}{\partial y}
		\end{array}
		\right]
	\end{array}
\end{equation}
where:
\begin{itemize}
	\item $\bar{\mu}$ is the depth-averaged viscosity
	\item $H$ is the ice thickness
	\item $\alpha$ is the basal friction coefficient
\end{itemize}

\paragraph{Boundary conditions}
At the surface of the ice sheet, $\Gamma_s$, we assume a stress-free boundary condition. A viscous friction law is applied at the base of the ice sheet, $\Gamma_b$, and water pressure is applied at the ice/water interface $\Gamma_w$. For FS, these boundary conditions are:
\begin{equation}
	\begin{array}{rcll}
		\boldsymbol{\sigma}\cdot {\bf n} & = & \boldsymbol{0} & \text{ on } \Gamma_s\\
		\\
		\left(\boldsymbol{\sigma}\cdot {\bf n} \cdot {\bf n} + {\alpha}^2 {\bf v}\right)_{\parallel} & = & \boldsymbol{0} & \text{ on } \Gamma_b\\
		\\
		{\bf v} \cdot {\bf n}  & = & 0 & \text{ on } \Gamma_b\\
		\\
		\boldsymbol{\sigma}\cdot {\bf n} & = & \rho_w g z {\bf n} & \text{ on } \Gamma_w
	\end{array}
\end{equation}
where
\begin{itemize}
	\item ${\bf n}$ is the outward-pointing unit normal vector
	\item $\rho_w$ is the water density
	\item $z$ is the vertical coordinate with respect to sea level
\end{itemize}

For HO, these boundary conditions become:
\begin{equation}
	\begin{array}{rclrcll}
		\dot{\boldsymbol{\varepsilon}}_{HO1} \cdot {\bf n} & = & 0 &
		\dot{\boldsymbol{\varepsilon}}_{HO2} \cdot {\bf n} & = & 0 &
		\text{ on } \Gamma_s\\
		\\
		2\mu\,\dot{\boldsymbol{\varepsilon}}_{HO1} \cdot {\bf n} & = & -\alpha^2 v_x &
		2\mu\,\dot{\boldsymbol{\varepsilon}}_{HO2} \cdot {\bf n} & = & -\alpha^2 v_y &
		\text{ on } \Gamma_b\\
		\\
		2\mu\,\dot{\boldsymbol{\varepsilon}}_{HO1} \cdot {\bf n} & = & f_w n_x &
		2\mu\,\dot{\boldsymbol{\varepsilon}}_{HO2} \cdot {\bf n} & = & f_w n_y &
		\text{ on } \Gamma_w
	\end{array}
\end{equation}
where $f_w=\rho g\left(s-z\right) +\rho_w g \min\left(z,0\right)$.

For SSA, these boundary conditions are:
\begin{equation}
	\begin{array}{rclrcll}
		\dot{\boldsymbol{\varepsilon}}_{SSA1} \cdot {\bf n} & = & 0 &
		\dot{\boldsymbol{\varepsilon}}_{SSA2} \cdot {\bf n} & = & 0 &
		\text{ on } \Gamma_s
	\end{array}
\end{equation}
\begin{equation}
	\begin{array}{rcl}
		2\bar{\mu}H\dot{\boldsymbol{\varepsilon}}_{SSA1} \cdot {\bf n} & = & \left(\frac{1}{2}\rho g H^2 - \frac{1}{2}\rho_w g b^2 \right) n_x \\
		\\
		2\bar{\mu}H\dot{\boldsymbol{\varepsilon}}_{SSA2} \cdot {\bf n} & = & \left(\frac{1}{2}\rho g H^2 - \frac{1}{2}\rho_w g b^2 \right) n_y
	\end{array}
	\text{ on } \Gamma_w
\end{equation}

\subsubsection{Model parameters}
The parameters relevant to the stress balance solution can be displayed by typing:
\begin{lstlisting}
>> md.stressbalance
\end{lstlisting}

\begin{itemize}
	\item \lstinlinebg|md.stressbalance.restol|: mechanical equilibrium residue convergence criterion
	\item \lstinlinebg|md.stressbalance.reltol|: velocity relative convergence criterion, (\lstinlinebg|NaN| if not applied)
	\item \lstinlinebg|md.stressbalance.abstol|: velocity absolute convergence criterion, (\lstinlinebg|NaN| if not applied)
	\item \lstinlinebg|md.stressbalance.maxiter|: maximum number of nonlinear iterations (default is 100)
	\item \lstinlinebg|md.stressbalance.spcvx|: x-axis velocity constraint (\lstinlinebg|NaN| means no constraint)
	\item \lstinlinebg|md.stressbalance.spcvy|: y-axis velocity constraint (\lstinlinebg|NaN| means no constraint)
	\item \lstinlinebg|md.stressbalance.spcvz|: z-axis velocity constraint (\lstinlinebg|NaN| means no constraint)
	\item \lstinlinebg|md.stressbalance.rift_penalty_threshold|: threshold for instability of mechanical constraints
	\item \lstinlinebg|md.stressbalance.rift_penalty_lock|: number of iterations before rift penalties are locked
	\item \lstinlinebg|md.stressbalance.penalty_factor|: offset used by penalties:
\end{itemize}
\begin{equation}
	\kappa=10^{\text{penalty\_factor}} \max_{i,j}\left| K_{ij}\right|
\end{equation}
\begin{itemize}
	\item \lstinlinebg|md.stressbalance.vertex_pairing|: pairs of vertices that are penalized
	\item \lstinlinebg|md.stressbalance.shelf_dampening|: use dampening for floating ice? Only for Stokes model
	\item \lstinlinebg|md.stressbalance.referential|: local referential
	\item \lstinlinebg|md.stressbalance.requested_outputs|: additional outputs requested
\end{itemize}

The solution will also use the following model fields:
\begin{itemize}
	\item \lstinlinebg|md.flowequations|: FS, HO or SSA
	\item \lstinlinebg|md.materials|: material parameters
	\item \lstinlinebg|md.initialization.vx|: x component of velocity (used as an initial guess)
	\item \lstinlinebg|md.initialization.vy|: y component of velocity (used as an initial guess)
	\item \lstinlinebg|md.initialization.vz|: y component of velocity (used as an initial guess)
\end{itemize}

\subsubsection{Running a simulation}
To run a simulation, use the following command:
\begin{lstlisting}
>> md = solve(md, 'Stressbalance');
\end{lstlisting}
The first argument is the model, the second is the nature of the simulation one wants to run.

\clearpage % Make sure all figures are placed before next section
