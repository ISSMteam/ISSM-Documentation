% Just The Docs Front Matter
% title: Interpolation Routines
% parent: Capabilities
% has_children: false
% has_toc: false

\subsection{Interpolation Routines} \label{sec:using-issm-capabilities-interpolation}
Several interpolation routines can be used in order to interpolate datasets onto the mesh vertices.

\lstinlinebg|ContourToMesh| is used to flag the nodes and/or elements that are within a contour from an Argus contour and a mesh. For example,
\begin{lstlisting}
gridinsidefront = ContourToMesh(md.mesh.elements, md.mesh.x, md.mesh.y, expread('Front.exp', 1), 'node');
\end{lstlisting}

To interpolate a field from a structured grid to an unstructured mesh (or any list of points), one can use \lstinlinebg|InterpFromGridToMesh|,
\begin{lstlisting}
data_mesh = InterpFromGridToMesh(x_grid, y_grid, data, x_mesh, y_mesh)
\end{lstlisting}

To interpolate a field from a 2d mesh to a 2d mesh (or any list of points), one can use \lstinlinebg|InterpFromMeshToMesh2d|,
\begin{lstlisting}
data_mesh2 = InterpFromMeshToMesh2d(index_mesh1, x_mesh1, y_mesh1, data, x_mesh2, y_mesh2)
\end{lstlisting}

To interpolate a field from a 3d mesh to a 3d mesh (or any list of points), one can use \lstinlinebg|InterpFromMeshToMesh3d|,
\begin{lstlisting}
data_mesh2 = InterpFromMeshToMesh3d(index_mesh1, x_mesh1, y_mesh1, z_mesh1, data, x_mesh2, y_mesh2, z_mesh2)
\end{lstlisting}

\clearpage % Make sure all figures are placed before next section
