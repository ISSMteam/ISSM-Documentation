% Just The Docs Front Matter
% title: Mass Transport Solution
% parent: Capabilities
% has_children: false
% has_toc: false

\subsection{Mass Transport Solution} \label{sec:using-issm-capabilities-mass-transport}

\subsubsection{Physical basis}
\paragraph{Conservation of mass}
The mass transport equation is derived from the depth-integrated form of the mass conservation equation and reads:
\begin{equation}
	\frac{\partial H}{\partial t} = - \nabla \cdot \left(H \bar{\bf v}\right) + \dot{M}_s - \dot{M}_b
\end{equation}
where:
\begin{itemize}
	\item $\bar{\bf v}$ is the depth-averaged velocity vector
	\item $H$ is the ice thickness
	\item $\dot{M}_s$ is the surface accumulation (in m/yr of ice equivalent, positive for accumulation)
	\item $\dot{M}_b$ is the basal melting (in m/yr of ice equivalent, positive for melting)
\end{itemize}

For full-Stokes models, free surface equations are solved for the upper surface and the base of floating ice:
\begin{equation}
	\frac{\partial s}{\partial t}
	+ v_x\left(s\right) \dfrac{\partial s}{\partial x}
	+ v_y\left(s\right) \dfrac{\partial s}{\partial y}
	- v_z\left(s\right)
	= \dot{M}_s
\end{equation}
and:
\begin{equation}
	\frac{\partial b}{\partial t}
	+ v_x\left(b\right) \dfrac{\partial b}{\partial x}
	+ v_y\left(b\right) \dfrac{\partial b}{\partial y}
	- v_z\left(b\right)
	= \dot{M}_b
\end{equation}
where:
\begin{itemize}
	\item $s$ is the elevation of the ice upper surface
	\item $b$ is the elevation of the floating ice lower surface
	\item $\left(v_x\left(s\right),v_y\left(s\right),v_z\left(s\right)\right)$ are the ice velocity
		components at the upper surface $s$
	\item $\left(v_x\left(b\right),v_y\left(b\right),v_z\left(b\right)\right)$ are the ice velocity
		components at the base $b$
\end{itemize}

\paragraph{Boundary conditions}
Ice thickness is imposed at the inflow boundary:
\begin{equation}
	H=H_{obs} \text{ on } \Gamma_{-}
\end{equation}

For free surfaces models, both $b$ and $s$ are constrained at the inflow boundary.

\paragraph{Numerical implementation}
Mass transport is solved using finite elements in space, and implicit finite difference in time. To stabilize the equation, a stabilization term might be added to the left hand side, for example:
\begin{equation}
	\frac{\partial H}{\partial t} + \nabla \cdot \left(H \bar{\bf v}\right) 
	- {\color{red} \nabla \cdot \left(\mathfrak{D} \nabla H\right)}
	= \dot{M}_s - \dot{M}_b
\end{equation}
where $\mathfrak{D}$ is the artificial diffusivity. We take:
\begin{equation}
	\mathfrak{D} = \frac{h}{2}
	\left(\begin{array}{cc}
		\left|vx\right| & 0 \\
		\\
		0 & \left|vy\right|
	\end{array}\right)
\end{equation}

There are other stabilization schemes available in ISSM: (1) Artificial Diffusion, (2) Streamline Upwinding, (3) Discontinuous Galerkin (DG), (4) Flux Corrected Transport (FCT), and (5) Streamline Upwind Petrov-Galerlin (SUPG). They can used by setting:
\begin{lstlisting}
>> md.masstransport.stabilization = 1;
\end{lstlisting}

\subsubsection{Model parameters}
The parameters relevant to the mass transport solution can be displayed by running:
\begin{lstlisting}
>> md.masstransport
\end{lstlisting}

\begin{itemize}
	\item \lstinlinebg|md.masstransport.spcthickness|: thickness constraints (\lstinlinebg|NaN| means no constraint)
	\item \lstinlinebg|md.masstransport.hydrostatic_adjustment|: adjustment of ice shelves upper and lower
		surfaces: \lstinlinebg|'Incremental'| or \lstinlinebg|'Absolute'|
	\item \lstinlinebg|md.masstransport.stabilization|: 0: no stabilization, 1: artificial diffusion, 2: streamline upwinding 3:
		discontinuous Galerkin, 4: flux corrected transport (FCT), 5: streamline upwind Petrov-Galerkin (SUPG)
	\item \lstinlinebg|md.masstransport.penalty_factor|: offset used by penalties
\end{itemize}
\begin{equation}
	\kappa=10^{\text{penalty\_offset}} \max_{i,j}\left| K_{ij}\right|
\end{equation}
\begin{itemize}
	\item \lstinlinebg|md.masstransport.vertex_pairing|: pairs of vertices that are penalized (for periodic
		boundary conditions only)
\end{itemize}

The solution will also use the following model fields:
\begin{itemize}
	\item \lstinlinebg|md.smb.ablation_rate|: surface ablation rate (in meters)
	\item \lstinlinebg|md.smb.mass_balance|: surface mass balance (in meters)
	\item \lstinlinebg|md.initialization.vx|: x component of velocity
	\item \lstinlinebg|md.initialization.vy|: y component of velocity
	\item \lstinlinebg|md.basalforcings.groundedice_melting_rate|: basal melting rate applied on grounded ice (positive if melting)
	\item \lstinlinebg|md.basalforcings.floatingice_melting_rate|: basal melting rate applied on floating ice (positive if melting)
	\item \lstinlinebg|md.smb.mass_balance|: surface mass balance (in meters/year ice equivalent)
	\item \lstinlinebg|md.timestepping.time_step|: length of time steps (in years)
\end{itemize}

\subsubsection{Running a simulation}
To run a simulation, use the following command:
\begin{lstlisting}
>> md = solve(md,'Masstransport');
\end{lstlisting}
The first argument is the model, the second is the nature of the simulation one wants to run. This will compute one time step of the mass transport equation; use the 
%__@LATEX_ONLY_START@__
\hyperref[sec:using-issm-capabilities-transient]{transient solution}
%__@LATEX_ONLY_END@__
%__@MARKDOWN_ONLY_START@__
% <a href="transient">transient solution</a>
%__@MARKDOWN_ONLY_END@__
for multiple time steps.

\clearpage % Make sure all figures are placed before next section
