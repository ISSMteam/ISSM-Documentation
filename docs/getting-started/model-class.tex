% Just The Docs Front Matter
% title: Model Class
% parent: Getting Started
% nav_order: 2
% has_children: false
% has_toc: false

\section{Model Class} \label{sec:getting-started-model-class}
All the data belonging to a model (geometry, node coordinates, results, etc.) is held in the same object \lstinlinebg|model|. To create a new model in MATLAB, run,
\begin{lstlisting}
md = model
\end{lstlisting}
and in Python,
\begin{lstlisting}
from model import *
md = model()
\end{lstlisting}
This will create a new model named \lstinlinebg|md| whose class is \lstinlinebg|model|. The information contained in the model \lstinlinebg|md| is grouped by class, each of which are comprised of fields related to that particular aspect of the model (e.g. mesh, material properties, friction, stressbalance solution, solution results). When one creates a new model, all of these fields are empty or \lstinlinebg|NaN| (not a number), but \lstinlinebg|md| is ready to be used as a model. The list of these classes is displayed by running, in MATLAB,

\begin{lstlisting}
>> md
md = 

               mesh: [1x1 mesh2d]           -- mesh properties
               mask: [1x1 mask]             -- defines grounded and floating elements
           geometry: [1x1 geometry]         -- surface elevation, bedrock topography, ice thickness,...
                     [...]
            results: [1x1 struct]           -- model results
       radaroverlay: [1x1 radaroverlay]     -- radar image for plot overlay
      miscellaneous: [1x1 miscellaneous]    -- miscellaneous fields
\end{lstlisting}
or, in Python,
\begin{lstlisting}
>>> print(md)
               mesh: [1x1 mesh2d]           -- mesh properties
               mask: [1x1 mask]             -- defines grounded and floating elements
           geometry: [1x1 geometry]         -- surface elevation, bedrock topography, ice thickness,...
                     [...]
       radaroverlay: [1x1 radaroverlay]      -- radar image for plot overlay
      miscellaneous: [1x1 miscellaneous]     -- miscellaneous fields
  stochasticforcing: [1x1 stochasticforcing] -- stochasticity applied to model forcings
\end{lstlisting}

Likewise, you can display all the fields associated with, for example, the model's mesh by running,
\begin{lstlisting}
>> md.mesh
\end{lstlisting}
or,
\begin{lstlisting}
>>> print(md.mesh)
\end{lstlisting}

\subsection{Saving/Loading a Model}
You can save the model with all its fields so that the saved file contains all of the information in the model by running,
\begin{lstlisting}
save squaremodel md
\end{lstlisting}
This will create a file \lstinlinebg|squaremodel.mat| made from the model \lstinlinebg|md|. Likewise, to load this file, run,
\begin{lstlisting}
>> loadmodel squaremodel
\end{lstlisting}
The loaded model will be named \lstinlinebg|md|.

\clearpage % Make sure all figures are placed before next section
