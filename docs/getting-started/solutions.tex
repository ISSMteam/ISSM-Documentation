% Just The Docs Front Matter
% title: Solutions
% parent: Getting Started
% nav_order: 3
% has_children: true
% has_toc: false

\section{Solutions} \label{sec:getting-started-solutions}
After parameterizing your model, you are ready to request a solution. In MATLAB, this is done by running,
\begin{lstlisting}
>> md = solve(md, <solution_type>);
\end{lstlisting}
and in Python,
\begin{lstlisting}
>>> md = solve(md, <solution_type>)
\end{lstlisting}
where \lstinlinebg|<solution_type>| is a string representing a given solution type, for example, \lstinlinebg|'Stressbalance'|.

The following pages provide more in-depth information on the various solution types,
%__@LATEX_ONLY_START@__
\begin{itemize}
	\item \hyperref[sec:using-issm-capabilities-thermal]{thermal}
	\item \hyperref[sec:using-issm-capabilities-hydrology]{hydrology}
	\item \hyperref[sec:using-issm-capabilities-stress-balance]{stress-balance}
	\item \hyperref[sec:using-issm-capabilities-mass-transport]{mass transport}
	\item \hyperref[sec:using-issm-capabilities-gia]{Glacial Isostatic Adjustment (GIA)}
\end{itemize}
%__@LATEX_ONLY_END@__
%__@MARKDOWN_ONLY_START@__
%- <a href="../using-issm/capabilities/thermal" target="_top">thermal</a>
%- <a href="../using-issm/capabilities/hydrology" target="_top">hydrology</a>
%- <a href="../using-issm/capabilities/stress-balance" target="_top">stress-balance</a>
%- <a href="../using-issm/capabilities/mass-transport" target="_top">mass transport</a>
%- <a href="../using-issm/capabilities/gia" target="_top">Glacial Isostatic Adjustment (GIA)</a>
%__@MARKDOWN_ONLY_END@__

Running one or more of the above solutions over time is detailed on the 
%__@LATEX_ONLY_START@__
\hyperref[sec:capabilities-transient]{Transient Solutions page}.
%__@LATEX_ONLY_END@__
%__@MARKDOWN_ONLY_START@__
%<a href="../using-issm/capabilities/transient" target="_top">Transient Solutions page</a>.
%__@MARKDOWN_ONLY_END@__

\clearpage % Make sure all figures are placed before next section
