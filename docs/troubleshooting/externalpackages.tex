% Just The Docs Front Matter
% title: Configuring and Compiling External Packages
% parent: Troubleshooting
% nav_order: 1
% has_children: false
% has_toc: true

\section{Configuring and Compiling External Packages} \label{sec:troubleshooting-compiling-externalpackages}
\subsection*{Chaco Multilevel Graph Partitioning Tool}
\subsubsection{Error compiling on macOS}%{{{
When compiling \href{https://github.com/sandialabs/Chaco}{Chaco} on macOS (most likely using the \lstinlinebg|${ISSM_DIR}/externalpackages/chaco/install.sh| script), you may encounter an error that reads,
\begin{lstlisting}
util/smalloc.c:6:10: fatal error: 'malloc.h' file not found
#include <malloc.h>
         ^~~~~~~~~~
1 error generated.
make: *** [util/smalloc.o] Error 1
\end{lstlisting}

To correct this, you will have to have either Xcode Command Line Tools (preferred) or Xcode installed.
\begin{itemize}
\item To install the Xcode Command Line Tools, run \lstinlinebg|xcode-select --install|
\item You can install Xcode through the Mac App Store
\end{itemize}

You will then need to modify the \lstinlinebg|CPATH| environment variable in your shell profile (e.g. \lstinlinebg|~/.bashrc|),
\begin{itemize}
\item If you installed the Xcode Command Line Tools, 
\begin{lstlisting}
export CPATH="/Library/Developer/CommandLineTools/SDKs/MacOSX.sdk/usr/include/malloc:/usr/include"
\end{lstlisting}
\item If you installed Xcode,
\begin{lstlisting}
export CPATH="/Applications/Xcode.app/Contents/Developer/Platforms/MacOSX.platform/Developer/SDKs/MacOSX.sdk/usr/include/malloc:/usr/include"
\end{lstlisting}
\end{itemize}

Then, source your shell profile again and rerun the installation script.
%}}}

\subsection*{PETSc}
\subsection*{Error running make on <EXT\_PKG>}%{{{
When using PETSc to install an external package you may encounter a failure that reads,
\begin{lstlisting}
Error running make on <EXT_PKG>
\end{lstlisting}

Inspection of \lstinlinebg|src/configure.log| should reveal a line similiar to,
\begin{lstlisting}
Error running make on <EXT_PKG>: Could not execute "['<CMD_STRING>']":
\end{lstlisting}
This is usually not a very helpful error message. What you can do is copy \lstinlinebg|<CMD_STRING>| and run it manually on the command line to reveal and correct the underlying errors.
%}}}

\subsubsection{Error running configure on MPICH}%{{{
With GCC 10 and later, when using PETSc to install MPICH, you may encounter a failure that reads,
\begin{lstlisting}
Error running configure on MPICH
\end{lstlisting}

Inspection of \lstinlinebg|src/configure.log| should reveal,
\begin{lstlisting}
error: The Fortran compiler gfortran will not compile files that call
the same routine with arguments of different types.
\end{lstlisting}
One way to get around this is to add,
\begin{lstlisting}
--FFLAGS="-fallow-argument-mismatch"
\end{lstlisting}
to the list of configuration arguments.
%}}}

\subsection*{C++ error! MPI\_Finalize() could not be located! / Error running make; make install on MPICH}%{{{
When configuring and/or compiling MPICH via PETSc, it may fail with either of the above error messages. Inspection of \lstinlinebg|src/configure.log| should reveal something like,
\begin{lstlisting}
Checking for program /opt/homebrew/bin/mpicc...found
\end{lstlisting}
The solution here is one of the following,
\begin{enumerate}
\item Uninstall the MPI compiler set installed via package manager (in this example, Homebrew)
\item Make sure that the path to the desired non-MPI compiler set (e.g. \lstinlinebg|gcc|, \lstinlinebg|g++|) appears in the \lstinlinebg|PATH| environment variable before the path to the offending MPI compiler set
\item Remove the \lstinlinebg|--download-mpich=1| option from the PETSc configuration and instead supply the appropriate options to use the alternate MPI implementation
\end{enumerate}
%}}}

\clearpage % Make sure all figures are placed before next section
